% !TeX TS-program = xelatex
\documentclass[aspectratio=169, 12pt, xcolor=table]{beamer}
\usefonttheme{professionalfonts}
\usefonttheme{serif}
\usepackage[T1]{fontenc}
\usepackage{fontspec-xetex}

\usepackage{tikz}

\usepackage{adjustbox}
\usepackage{booktabs}
\usepackage{ifthen}
\usepackage{listings}
\usepackage{subcaption}

\usetikzlibrary{shapes.geometric, arrows}

\setmainfont{Lato}

%\PassOptionsToPackage[more=table]{xcolor}

% Local configuration
\renewcommand{\figurename}{}
\DeclareCaptionFormat{custom}
{%
	\tiny #3
}
\captionsetup{format=custom}

% Title stuff
\title{Games Technologies}
\subtitle{Inheritance}
\date{Week 03}
\author{Vassilis Markos, Mediterranean College}

\usetheme{streamline}

% Local Commands
\newcommand{\ohref}[1]{\href{#1}{\texttt{#1}}}
\newcommand{\listindex}[2]{{\underset{#1}{\small #2}}}

% Code listings

\definecolor{codegreen}{rgb}{0,0.6,0}
\definecolor{codegray}{rgb}{0.5,0.5,0.5}
\definecolor{codepurple}{rgb}{0.58,0,0.82}
\definecolor{backcolour}{rgb}{0.95,0.95,0.92}

\lstdefinestyle{mystyle}{
	backgroundcolor=\color{backcolour},   
	commentstyle=\color{codegreen},
	keywordstyle=\color{magenta},
	numberstyle=\tiny\color{codegray},
	stringstyle=\color{codepurple},
	basicstyle=\ttfamily\footnotesize,
	breakatwhitespace=false,         
	breaklines=true,                 
	captionpos=b,                    
	keepspaces=true,                 
	numbers=left,                    
	numbersep=5pt,                  
	showspaces=false,                
	showstringspaces=false,
	showtabs=false,                  
	tabsize=2
}

\lstset{style=mystyle}

% Tikz style

\tikzstyle{startstop} = [ellipse, rounded corners, minimum width=2cm, minimum height=0.8cm, text centered, draw=black, fill=none]
\tikzstyle{io} = [trapezium, trapezium left angle=70, trapezium right angle=110, minimum width=2cm, minimum height=0.8cm, text centered, draw=black, fill=none]
\tikzstyle{process} = [rectangle, minimum width=2cm, minimum height=0.8cm, text centered, draw=black, fill=none]
\tikzstyle{decision} = [diamond, minimum width=2cm, minimum height=0.8cm, text centered, draw=black, fill=none]
\tikzstyle{arrow} = [thick,->,>=stealth]

\tikzstyle{simple node} = [rectangle, draw=black, minimum width = 0.8cm, minimum height = 0.6cm]

% makeatletter stuff

\makeatletter
\newcommand{\arabicthree}[1]{\expandafter\@arabicthree\csname c@#1\endcsname}
\newcommand{\@arabicthree}[1]{\ifnum #1<100 0\fi\ifnum #1<10 0\fi\number#1}
\makeatother

\newcounter{exno}
\setcounter{exno}{0}

\newcommand{\exno}{\stepcounter{exno}In--class Exercise \#\arabicthree{exno}}

\begin{document}
	
	\begin{frame}
		\titlepage
	\end{frame}
	
	\begin{frame}{Contents}
		\tableofcontents
	\end{frame}
	
	\section{Inheritance}\label{sec:inheritance}
	
	\sectionframe
	
	\begin{frame}{On Animals}
		\begin{itemize}[<+->]
			\item Last time, we explored some simple Animal functionality, all wrapped in an \texttt{Animal} class.
			\begin{itemize}[<+->]
				\item As always, any code mentioned here lives in the \texttt{source} directory, in particular, in \texttt{Animal.cs}.
			\end{itemize}
			\item But, ``animal'' is a quite generic term, since various animals have different properties that distinguish them from others.
			\item For instance, many animals do ``talk'' in some sense, but, well not in the same way.
			\begin{itemize}[<+->]
				\item Cats ``meow'', dogs ``woof'' and so on.
				\item And some don't talk.
			\end{itemize}
			\item And, not to be forgotten, all animals share some characteristics too!
		\end{itemize}
	\end{frame}
	
	\begin{frame}{On Classes}
		\begin{itemize}[<+->]
			\item So, we need a way to compactly express all of these things in the OOP paradigm.
			\item Evidently, a simple approach would be to define a single class for each animal we are interested in.
			\item Can you see any problems here?
			\begin{itemize}[<+->]
				\item \emph{Redundancy,} to begin with.
				\item Lack of semantic mapping of those concepts in our design is another.
			\end{itemize}
			\item In order to facilitate this sort of sharing some attributes / functionality but differentiating with respect to some other, OOP offers \textbf{Inheritance.}
		\end{itemize}
	\end{frame}

	\begin{frame}{Examples Of Inheritance}
		\lstinputlisting[linerange={55-63}, language=C++]{../source/Animal.cs}
	\end{frame}

	\begin{frame}{Examples Of Inheritance}
		\lstinputlisting[linerange={1-6}, language=C++]{../source/Cat.cs}
	\end{frame}

	\begin{frame}{Examples Of Inheritance}
		\lstinputlisting[linerange={8-14}, language=C++]{../source/Cat.cs}
	\end{frame}
	
	\begin{frame}{Examples Of Inheritance}
		\lstinputlisting[linerange={16-21}, language=C++]{../source/Cat.cs}
		\begin{itemize}[<+->]
			\item When it comes to overriding, we just need to define a child method with the same signature as the parent method.
			\item The parent method has to be declared as \texttt{virtual}.
			\item The child method has to be declared as \texttt{override}.
		\end{itemize}
	\end{frame}

	\begin{frame}{A Simple Test Class}
		\lstinputlisting[language=C++]{../source/Test.cs}
	\end{frame}
	
	\begin{frame}{Stranger Things}
		\lstinputlisting[language=C++]{../source/StrangeTest.cs}
	\end{frame}

	\begin{frame}{Types And Inheritance}
		\begin{itemize}[<+->]
			\item We can construct an object using a constructor of a child class and then assign it to a variable of a parent class.
			\item This is perfectly okay, and in many cases useful (examples to come soon).
			\item However, since the object has been constructed with some child constructor, it is considered an object of type \texttt{child} and not \texttt{parent}.
			\item Thus, above, in both cases, the two instances use the overriden version of the \texttt{Talk()} method.
		\end{itemize}
	\end{frame}

	\begin{frame}{Dogs}
		\lstinputlisting[language=C++]{../source/Dog.cs}
	\end{frame}

	\begin{frame}{Animal Structures}
		\lstinputlisting[language=C++, basicstyle={\ttfamily\scriptsize}]{../source/AnimalStructures.cs}
	\end{frame}

	\begin{frame}{Types And Inheritance (Again)}
		\begin{itemize}[<+->]
			\item Using inheritance one can simulate a higher level behaviour found, e.g., in Python, where a data structure can contain elements of different type.
			\item This is naturally restricted to elements of some common ancestor type, in our case \texttt{Animal}.
			\item However, this is enough in most cases to allow for significant code simplifications, while still maintaining the advantages of strong typing.
			\item This is also useful in function / method signatures, in case one needs to handle multiple child types in a uniform way.
		\end{itemize}
	\end{frame}
	
	\section{Fun Time!}\label{sec:fun-time}
	
	\sectionframe
	
	\setcounter{exno}{0}
	
	\begin{frame}[fragile]{\exno}
		Follow Lab instructions in
		\begin{center}
			\texttt{lab/Game\_Lab\_01.pdf}
		\end{center}
		Use any online resources you might find useful.
	\end{frame}

	\begin{frame}[fragile]{\exno}
		Follow Lab instructions in
		\begin{center}
			\texttt{lab/Game\_Lab\_02.pdf}
		\end{center}
		Use any online resources you might find useful.
	\end{frame}
	
	\begin{frame}{Homework}
		Complete any incomplete lab exercises and then proceed to complete any missing parts of the game lab discussed in class today.
	\end{frame}
	
	\begin{frame}{Any Questions?}
		\begin{minipage}{0.35\textwidth}
			\raggedright
			Do not forget to fill in the questionnaire shown right!
		\end{minipage}\hfill
		\begin{minipage}{0.58\textwidth}
			\vspace{0pt}
			\raggedleft
			\includegraphics[scale=0.4]{../../assets/post_lesson_assessment.png}
			\centering
			\ohref{https://forms.gle/dKSrmE1VRVWqxBGZA}
		\end{minipage}
	\end{frame}
	
\end{document}