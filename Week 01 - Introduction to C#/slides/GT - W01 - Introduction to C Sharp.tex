% !TeX TS-program = xelatex
\documentclass[aspectratio=169, 12pt, xcolor=table]{beamer}
\usefonttheme{professionalfonts}
\usefonttheme{serif}
\usepackage[T1]{fontenc}
\usepackage{fontspec-xetex}

\usepackage{tikz}

\usepackage{adjustbox}
\usepackage{booktabs}
\usepackage{ifthen}
\usepackage{listings}
\usepackage{subcaption}

\usetikzlibrary{shapes.geometric, arrows}

\setmainfont{Lato}

%\PassOptionsToPackage[more=table]{xcolor}

% Local configuration
\renewcommand{\figurename}{}
\DeclareCaptionFormat{custom}
{%
	\tiny #3
}
\captionsetup{format=custom}

% Title stuff
\title{Games Technologies}
\subtitle{Introduction to C\#}
\date{Week 01}
\author{Vassilis Markos, Mediterranean College}

\usetheme{streamline}

% Local Commands
\newcommand{\ohref}[1]{\href{#1}{\texttt{#1}}}
\newcommand{\listindex}[2]{{\underset{#1}{\small #2}}}

% Code listings

\definecolor{codegreen}{rgb}{0,0.6,0}
\definecolor{codegray}{rgb}{0.5,0.5,0.5}
\definecolor{codepurple}{rgb}{0.58,0,0.82}
\definecolor{backcolour}{rgb}{0.95,0.95,0.92}

\lstdefinestyle{mystyle}{
	backgroundcolor=\color{backcolour},   
	commentstyle=\color{codegreen},
	keywordstyle=\color{magenta},
	numberstyle=\tiny\color{codegray},
	stringstyle=\color{codepurple},
	basicstyle=\ttfamily\footnotesize,
	breakatwhitespace=false,         
	breaklines=true,                 
	captionpos=b,                    
	keepspaces=true,                 
	numbers=left,                    
	numbersep=5pt,                  
	showspaces=false,                
	showstringspaces=false,
	showtabs=false,                  
	tabsize=2
}

\lstset{style=mystyle}

% Tikz style

\tikzstyle{startstop} = [ellipse, rounded corners, minimum width=2cm, minimum height=0.8cm, text centered, draw=black, fill=none]
\tikzstyle{io} = [trapezium, trapezium left angle=70, trapezium right angle=110, minimum width=2cm, minimum height=0.8cm, text centered, draw=black, fill=none]
\tikzstyle{process} = [rectangle, minimum width=2cm, minimum height=0.8cm, text centered, draw=black, fill=none]
\tikzstyle{decision} = [diamond, minimum width=2cm, minimum height=0.8cm, text centered, draw=black, fill=none]
\tikzstyle{arrow} = [thick,->,>=stealth]

\tikzstyle{simple node} = [rectangle, draw=black, minimum width = 0.8cm, minimum height = 0.6cm]

% makeatletter stuff

\makeatletter
\newcommand{\arabicthree}[1]{\expandafter\@arabicthree\csname c@#1\endcsname}
\newcommand{\@arabicthree}[1]{\ifnum #1<100 0\fi\ifnum #1<10 0\fi\number#1}
\makeatother

\newcounter{exno}
\setcounter{exno}{0}

\newcommand{\exno}{\stepcounter{exno}In--class Exercise \#\arabicthree{exno}}

\begin{document}
	
	\begin{frame}
		\titlepage
	\end{frame}
	
	\begin{frame}{Contents}
		\tableofcontents
	\end{frame}
	
	\section{Key C\# Constructs}\label{sec:key-cshar-constructs}
	
	\sectionframe
		
	\begin{frame}{Hello, World!}
		\lstinputlisting[language=C++]{../source/helloWorld.cs}
	\end{frame}

	\begin{frame}{The Anatomy of a C\# Program}
		\begin{itemize}
			\item C\# tends to be more verbose than Python in most cases.
			\item To begin with, everything should be wrapped in \texttt{namespace}s, i.e., collections of entities that are conceptually grouped together.
			\item Then, almost everything lives within a class, much like Java and other object-oriented languages.
			\item Finally, each project should have a single \textbf{main entry point}, which is a function called \texttt{Main()} which is typically typed as \texttt{static} and \texttt{void} (more on that soon\ldots).
		\end{itemize}
	\end{frame}

	\begin{frame}{Static Typing}
		\begin{adjustbox}{width=1.3\textwidth}
			\lstinputlisting[language=C++]{../source/staticTyping.cs}
		\end{adjustbox}
	\end{frame}

	\begin{frame}{Static Typing}
		\begin{itemize}
			\item In Python, variables are really flexible, being capable of pointing to virtually any language construct.
			\item In C\#, things are a bit more strict, inheriting some C-style semantics.
			\item Variable types have to be declared explicitly upon variable declaration, determining the size in memory a variable occupies and some other things under the hood.
			\item One can use the wildcard \texttt{var} to let the compiler guess, which, in general, is not a good idea, and should be avoided as much as possible.
			\begin{itemize}
				\item However, it is quite convenient to have a third party badly documented API endpoint cast this way\ldots
			\end{itemize}
		\end{itemize}
	\end{frame}

	\begin{frame}{Flow Control}
		\begin{adjustbox}{width=0.7\textwidth}
			\lstinputlisting[language=C++]{../source/flowControl.cs}
		\end{adjustbox}
	\end{frame}
	
	\begin{frame}{Flow Control}
		\begin{itemize}
			\item Besides, syntax, the same common flow control constructs can be used in C\#.
			\item \textbf{Beware:} \texttt{else if}, not \texttt{elif}!
			\item Also, deriving from C-style syntax, the logic operators are:
			\begin{itemize}
				\item \texttt{and}: \texttt{\&\&}
				\item \texttt{or}: \texttt{||}
				\item \texttt{not}: \texttt{!}
			\end{itemize}
		\end{itemize}
	\end{frame}

	\begin{frame}{While Loops}
		\begin{adjustbox}{width=0.7\textwidth}
			\lstinputlisting[language=C++]{../source/whileLoop.cs}
		\end{adjustbox}
	\end{frame}

	\begin{frame}{For Loops}
		\begin{adjustbox}{width=1.55\textwidth}
			\lstinputlisting[language=C++]{../source/forLoop.cs}
		\end{adjustbox}
	\end{frame}

	\begin{frame}{Functions}
		\begin{adjustbox}{width=0.65\textwidth}
			\lstinputlisting[language=C++]{../source/function.cs}
		\end{adjustbox}
	\end{frame}

	\begin{frame}{Function Signatures}
		\begin{itemize}
			\item In contrast to Python, a function is not identified solely by its name.
			\item In C\#, each function has its unique function signature, which comprises of:
			\begin{itemize}
				\item its name;
				\item its number and type(s) of arguments;
				\item its return type.
			\end{itemize}
			\item So, you can define two functions with the same name, but at least one of the above attributes being different.
			\item This allow for the so-called \textbf{function overloading} (more on that soon).
		\end{itemize}
	\end{frame}

	\begin{frame}{Common Data Structures}
		\begin{adjustbox}{width=\textwidth}
			\lstinputlisting[language=C++]{../source/dataStructures.cs}
		\end{adjustbox}
	\end{frame}
	
	\section{Fun Time!}\label{sec:fun-time}
	
	\sectionframe
	
	\setcounter{exno}{0}
	
	\begin{frame}[fragile]{\exno}
		Make all transcriptions from Python to C\# found in:
		\begin{center}
			\texttt{lab/transcriptions.pdf}
		\end{center}
		Use any only resources you might find useful.
	\end{frame}
	
	\begin{frame}{Homework}
		Complete any incomplete lab exercises and then proceed to complete any missing parts of the game lab discussed in class today.
	\end{frame}
	
	\begin{frame}{Any Questions?}
		\begin{minipage}{0.35\textwidth}
			\raggedright
			Do not forget to fill in the questionnaire shown right!
		\end{minipage}\hfill
		\begin{minipage}{0.58\textwidth}
			\vspace{0pt}
			\raggedleft
			\includegraphics[scale=0.4]{../../assets/post_lesson_assessment.png}
			\centering
			\ohref{https://forms.gle/dKSrmE1VRVWqxBGZA}
		\end{minipage}
	\end{frame}
	
\end{document}