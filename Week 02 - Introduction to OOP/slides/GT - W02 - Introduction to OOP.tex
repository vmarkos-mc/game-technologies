% !TeX TS-program = xelatex
\documentclass[aspectratio=169, 12pt, xcolor=table]{beamer}
\usefonttheme{professionalfonts}
\usefonttheme{serif}
\usepackage[T1]{fontenc}
\usepackage{fontspec-xetex}

\usepackage{tikz}

\usepackage{adjustbox}
\usepackage{booktabs}
\usepackage{ifthen}
\usepackage{listings}
\usepackage{subcaption}

\usetikzlibrary{shapes.geometric, arrows}

\setmainfont{Lato}

%\PassOptionsToPackage[more=table]{xcolor}

% Local configuration
\renewcommand{\figurename}{}
\DeclareCaptionFormat{custom}
{%
	\tiny #3
}
\captionsetup{format=custom}

% Title stuff
\title{Games Technologies}
\subtitle{Introduction to OOP}
\date{Week 02}
\author{Vassilis Markos, Mediterranean College}

\usetheme{streamline}

% Local Commands
\newcommand{\ohref}[1]{\href{#1}{\texttt{#1}}}
\newcommand{\listindex}[2]{{\underset{#1}{\small #2}}}

% Code listings

\definecolor{codegreen}{rgb}{0,0.6,0}
\definecolor{codegray}{rgb}{0.5,0.5,0.5}
\definecolor{codepurple}{rgb}{0.58,0,0.82}
\definecolor{backcolour}{rgb}{0.95,0.95,0.92}

\lstdefinestyle{mystyle}{
	backgroundcolor=\color{backcolour},   
	commentstyle=\color{codegreen},
	keywordstyle=\color{magenta},
	numberstyle=\tiny\color{codegray},
	stringstyle=\color{codepurple},
	basicstyle=\ttfamily\footnotesize,
	breakatwhitespace=false,         
	breaklines=true,                 
	captionpos=b,                    
	keepspaces=true,                 
	numbers=left,                    
	numbersep=5pt,                  
	showspaces=false,                
	showstringspaces=false,
	showtabs=false,                  
	tabsize=2
}

\lstset{style=mystyle}

% Tikz style

\tikzstyle{startstop} = [ellipse, rounded corners, minimum width=2cm, minimum height=0.8cm, text centered, draw=black, fill=none]
\tikzstyle{io} = [trapezium, trapezium left angle=70, trapezium right angle=110, minimum width=2cm, minimum height=0.8cm, text centered, draw=black, fill=none]
\tikzstyle{process} = [rectangle, minimum width=2cm, minimum height=0.8cm, text centered, draw=black, fill=none]
\tikzstyle{decision} = [diamond, minimum width=2cm, minimum height=0.8cm, text centered, draw=black, fill=none]
\tikzstyle{arrow} = [thick,->,>=stealth]

\tikzstyle{simple node} = [rectangle, draw=black, minimum width = 0.8cm, minimum height = 0.6cm]

% makeatletter stuff

\makeatletter
\newcommand{\arabicthree}[1]{\expandafter\@arabicthree\csname c@#1\endcsname}
\newcommand{\@arabicthree}[1]{\ifnum #1<100 0\fi\ifnum #1<10 0\fi\number#1}
\makeatother

\newcounter{exno}
\setcounter{exno}{0}

\newcommand{\exno}{\stepcounter{exno}In--class Exercise \#\arabicthree{exno}}

\begin{document}
	
	\begin{frame}
		\titlepage
	\end{frame}
	
	\begin{frame}{Contents}
		\tableofcontents
	\end{frame}
	
	\section{Object Oriented Programming}\label{sec:oop}
	
	\sectionframe
	
	\begin{frame}{Programmer Humour}
		\begin{figure}
			\includegraphics[width=0.75\textwidth]{assets/the_general_problem.png}
			\caption{This might make more sense after this set of slides.}
		\end{figure}
	\end{frame}
	
	
	\begin{frame}{What Is OOP?}
		Object Oriented Programming, as per Wikipedia (\href{https://en.wikipedia.org/wiki/Object-oriented_programming}{https://en.wikipedia.org/wiki/Object-oriented\_programming}):
		\begin{quote}
			Object-oriented programming (OOP) is a programming paradigm based on objects -- software entities that encapsulate data and function(s). An OOP computer program consists of objects that interact with one another.
		\end{quote}
		So, technically, data bound to actions taken with / about those data.
	\end{frame}

	\begin{frame}{Why OOP?}
		\begin{itemize}[<+->]
			\item Most of the times, OOP saves up tons of LOC (lines of code).
			\item In many cases, it is just natural to write in an OO way, since there is a clear mapping between the problem and classes / objects.
			\item It (typically) makes code readable and easily extensible.
			\item However, things come at a cost, especially in terms of a relatively steep learning curve.
		\end{itemize}
	\end{frame}

	\begin{frame}{OOP in C\#}
		\begin{itemize}[<+->]
			\item In C\# the key OOP construct is the \texttt{class}.
			\item You can think of classes as collections of:
			\begin{itemize}[<+->]
				\item tightly related data, alongside;
				\item relevant functions that operate on / with those data.
			\end{itemize}
			\item But, an example might be worth a thousand words in this case\ldots
		\end{itemize}
	\end{frame}

	\begin{frame}{C\# Classes}
		A C\# class consists of several sections:
		\begin{itemize}[<+->]
			\item The class fields, which correspond to the attributes of each object.
			\item The class methods, which are just functions related to the objects themselves.
			\item Typically, setters and getters for any private class fields.
		\end{itemize}
	\end{frame}

	\begin{frame}{Class Fields}
		\lstinputlisting[linerange={1-9}, language=C++]{../source/Animal.cs}
	\end{frame}

	\begin{frame}{Instance Constructor}
		\lstinputlisting[linerange={11-17}, language=C++]{../source/Animal.cs}
	\end{frame}
	
	\begin{frame}{More Constructors}
		\lstinputlisting[linerange={19-23}, language=C++]{../source/Animal.cs}
	\end{frame}

	\begin{frame}{Class Method}
		\lstinputlisting[linerange={25-29}, language=C++]{../source/Animal.cs}
	\end{frame}

	\begin{frame}{Class Setters / Getters}
		\resizebox{0.6\textwidth}{!}{
			\lstinputlisting[linerange={31-46}, language=C++]{../source/Animal.cs}
		}
	\end{frame}

	\begin{frame}{Method Overriding}
		\lstinputlisting[linerange={48-51}, language=C++]{../source/Animal.cs}
	\end{frame}

	\begin{frame}{Main Entry Point}
		\lstinputlisting[language=C++]{../source/Test.cs}
	\end{frame}
	
	
	\section{Fun Time!}\label{sec:fun-time}
	
	\sectionframe
	
	\setcounter{exno}{0}
	
	\begin{frame}[fragile]{\exno}
		Follow Lab instructions in
		\begin{center}
			\texttt{lab/Game\_Lab\_01.pdf}
		\end{center}
		Use any only resources you might find useful.
	\end{frame}
	
	\begin{frame}{Homework}
		Complete any incomplete lab exercises and then proceed to complete any missing parts of the game lab discussed in class today.
	\end{frame}
	
	\begin{frame}{Any Questions?}
		\begin{minipage}{0.35\textwidth}
			\raggedright
			Do not forget to fill in the questionnaire shown right!
		\end{minipage}\hfill
		\begin{minipage}{0.58\textwidth}
			\vspace{0pt}
			\raggedleft
			\includegraphics[scale=0.4]{../../assets/post_lesson_assessment.png}
			\centering
			\ohref{https://forms.gle/dKSrmE1VRVWqxBGZA}
		\end{minipage}
	\end{frame}
	
\end{document}