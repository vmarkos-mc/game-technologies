\documentclass[11pt]{article}
\usepackage[utf8]{inputenc}
\usepackage[margin=1in]{geometry}
\usepackage{listings}
\usepackage{xcolor}

\definecolor{codeblue}{rgb}{0,0,0.5}
\definecolor{codegray}{rgb}{0.5,0.5,0.5}

\lstset{
	basicstyle=\ttfamily\small,
	breaklines=true,
	frame=single,
	rulecolor=\color{codegray},
	numbers=left,
	numberstyle=\tiny\color{codegray}
}

\title{C\# Lab: Transferring your Python Knowledge}
\author{\small V. Markos}
\date{}

\begin{document}
	
	\maketitle
	
	\section*{Instructions}
	Translate the following Python functions into idiomatic C\#. Pay close attention to \textbf{types}, \textbf{access modifiers}, and \textbf{method naming conventions} (PascalCase).
	
	\subsection*{Task 1: Basic Multiplier}
	\textbf{Python:}
	\begin{lstlisting}[language=Python]
def multiply_numbers(a, b):
	return a * b
	\end{lstlisting}
	
	\textbf{C\# Hint:} Think about whether the numbers should be \texttt{int}, \texttt{float}, or \texttt{double}.
	
	\subsection*{Task 2: List Filtering}
	\textbf{Python:}
	\begin{lstlisting}[language=Python]
def get_long_names(name_list):
	result = []
	for name in name_list:
		if len(name) > 5:
			result.append(name)
	return result
	\end{lstlisting}
	
	\textbf{C\# Hint:} You will need \texttt{List<string>}. Try writing this with a \texttt{foreach} loop first, then see if you can use \texttt{LINQ} (\texttt{.Where()}).
	

	\subsection*{Task 3: Dictionary Search}
	\textbf{Python:}
	\begin{lstlisting}[language=Python]
def get_user_age(user_dict, name):
	if name in user_dict:
		return f"{name} is {user_dict[name]} years old"
	return "User not found"
	\end{lstlisting}
	
	\textbf{C\# Hint:} Use \texttt{Dictionary<string, int>}. Remember string interpolation starts with \texttt{\$}.
	
\end{document}